%%%%%%%%%%%%%%%%%%%%%%%%%%%%%%%%%%%%%%%%%
% Medium Length Professional CV
% LaTeX Template
% Version 2.0 (8/5/13)
%
% This template has been downloaded from:
% http://www.LaTeXTemplates.com
%
% Original author:
% Trey Hunner (http://www.treyhunner.com/)
%
% Important note:
% This template requires the resume.cls file to be in the same directory as the
% .tex file. The resume.cls file provides the resume style used for structuring the
% document.
%
%%%%%%%%%%%%%%%%%%%%%%%%%%%%%%%%%%%%%%%%%

%----------------------------------------------------------------------------------------
%	PACKAGES AND OTHER DOCUMENT CONFIGURATIONS
%----------------------------------------------------------------------------------------

\documentclass{resume} % Use the custom resume.cls style

\usepackage[left=0.75in,top=0.6in,right=0.75in,bottom=0.6in]{geometry} % Document margins
\title{Resume-Akshay-Chandrashekaran}

\name{Akshay Chandrashekaran} % Your name
\address{NASA Ames Research Park, Bldg.19, Moffett Field, CA 94035} % Your secondary addess (optional)
\address{web@akshayc.com} % Your phone number and email
\address{Carnegie Mellon University} % Your address



\begin{document}
\vspace{-2em}
%----------------------------------------------------------------------------------------
%	EDUCATION SECTION
%----------------------------------------------------------------------------------------

\begin{rSection}{Education}

{\bf Carnegie Mellon University}{\em Advisor: Prof. Ian Lane} \hfill {Moffett Field, CA}  \\
PhD. Candidate in Electrical and Computer Engineering, GPA: 3.64/4.0 \hfill {\em Jan 2012 - Present} \\
\\
{\bf Carnegie Mellon University} \hfill {Pittsburgh, PA} \\
M.S. in Electrical and Computer Engineering, GPA: 3.63/4.00 \hfill {\em Aug-2011- Dec 2011} \\
\\
{\bf Vishwakarma Institute of Technology} \hfill {Pune, India} \\
B.E. in Electronics and Telecommunication, CPA: 8.1/10  \hfill {\em Aug 2006 - May 2010}\\ 

{\bf Research Interests}{}\\
{My research interests include automated multi-objective optimization of hyper-parameters, and speech recognition on embedded platforms}
\end{rSection}


%----------------------------------------------------------------------------------------
%	WORK EXPERIENCE SECTION
%----------------------------------------------------------------------------------------

\begin{rSection}{Research and Work Experience}

\begin{rSubsection} {Capio Inc.} {Belmont, CA} {Speech Scientist} {June 2017 - Present}
\item Working as a speech scientist to accelerate optimization of neural network-based speech recognition models.
\item Working on rapid development of Speech recognition models for multiple low-resource languages.
\item Integration of Automated hyper-parameter optimization techniques for ASR model combination.
\item Exploratory work on online speaker change detection.
\item Integration of dynamic word addition to online speech recognition.
\item Integration of automated hyper-parameter optimization for decoder hyper-parameters.
\end{rSubsection}
\begin{rSubsection}{Carnegie Mellon University Silicon Valley}{Moffett Field, CA}{PhD. Candidate}{Jan 2012 - Present}
\item Currently working on Automated Multi-objective hyper-parameter optimization for speech recognition.
\item Working on development and analysis of methods to utilize validation curves from previous hyper-parameter configurations to predict the terminal performance of the current configuration.
\item Worked on development of a hierarchical optimization technique for feature, model and decoder hyper-parameters to jointly optimize towards word error rate and computational efficiency.
\item Worked on speech recognition for low resource languages.
\item Worked on Speech recognition on mobile and embedded platforms.
\item Developed Deep Neural Network Acoustic models for Android Platform using openCL.
\item Developed LSTM Acoustic model implementations using openCL.
\end{rSubsection}

%------------------------------------------------

\begin{rSubsection}{Baidu SVAIL}{Sunnyvale, CA}{Research Intern}{May 2016 - Aug 2016}
\item Worked on importance sampling-based data sampling techniques to improve training time for speech recognition.
\end{rSubsection}
%------------------------------------------------
\newpage
\begin{rSubsection}{Lenovo}{San Jose, CA}{Research Intern}{Oct 2013 - June 2014}
\item Developed a software framework for multi-modal interaction for applications in mobile devices.
\item Co-inventor in three resultant patents. 
\end{rSubsection}

%------------------------------------------------

\begin{rSubsection}{Carnegie Mellon University}{Pittsburgh, PA}{Graduate Assistant}{Jan 2011 - Dec 2011}
\item Researched on imagined speech classification using Electro-Encephalogram (EEG) signals
\end{rSubsection}

%------------------------------------------------

\begin{rSubsection}{Carnegie Mellon University}{Pittsburgh, PA}{Graduate Assistant}{Aug 2010 - Dec 2010}
\item Synaptic Bouton detection from images of visual cortex of a tree shrew. 
\end{rSubsection}
\end{rSection}

%------------------------------------------------
\begin{rSection}{Patents}
\begin{enumerate}
\item A. Raux, A. Chandrashekaran,``\emph{Multi-Modal Fusion Engine}'' (2014)
\item A. Raux, A. Chandrashekaran, ``\emph{Selecting Multimodal Elements}'' (2014)
\item A. Raux, A. Chandrashekaran, ``\emph{Identification of User Input Within an Application}"" (2014)
%\item 
\end{enumerate}

\end{rSection}

\begin{rSection}{Publications}
\begin{enumerate}
\item A. Chandrashekaran, I. Lane, ``\emph{Auto-ML for Automated Optimization of Speech Recognition on Mobile Devices}'', GTC 2018 (Poster)
\item K. Han, A. Chandrashekaran, J. Kim, I. Lane, ``\emph{Densely Connected Networks for Conversational Speech Recognition}'', Interspeech 2018 (Submitted)
\item K. Han, A. Chandrashekaran, J. Kim, I. Lane, ``\emph{The CAPIO 2017 Conversational Speech Recognition System}'', ArXiv Preprint \emph{1801.00059} 
\item A. Chandrashekaran, I. Lane ``\emph{Speeding up Hyper-parameter Optimization by Extrapolation of Learning Curves using Previous Builds}'', ECML 2017
\item A. Chandrashekaran, I. Lane, ``\emph{Hierarchical Constrained Bayesian Optimization for Feature, Acoustic Model and Decoder Parameter Optimization}'', Interspeech 2017.
\item  A. Chandrashekaran, I. Lane, ``\emph{Automated optimization of decoder hyper-parameters for online LVCSR}'', Spoken Language Technologies Workshop (SLT 2016).
\item  A. Chandrashekaran, I. Lane, ``\emph{Automated Feature and Model Optimization for Task-specific Acoustic Models}'', BayLearn 2015 (Poster).

\item D. Cohen, A. Chandrashekaran, I. Lane, A. Raux, ``\emph{The hri-cmu corpus of situated in-car interactions.}'', Proceedings for International Workshop Series on Spoken Dialogue Systems Technology (IWSDS 2014).
 
 \item I. Lane, V. Prasad, G. Sinha, A. Umuhoza, S. Luo, A. Chandrashekaran, A. Raux, ``\emph{HRItk: the human-robot interaction ToolKit rapid development of speech-centric interactive systems in ROS.}'' NAACL-HLT Workshop on Future Directions and Needs in the Spoken Dialog Community: Tools and Data (NAACL-HLT 2012). Association for Computational Linguistics.
 \end{enumerate}

%{\bf To be submitted}{}\\ 
%\vspace{-1.0em}
%\begin{enumerate}
 %\item Y. Wang and {\bf M. Zeng}, ``\emph{Improving Text Classification by Capturing Local Feature from Word Sequences}", SIAM International %Conference on Data Mining (SDM 2016)
 
% \item {\bf M. Zeng}, L. T. Nguyen, O. J. Mengshoel and J. Zhang ``\emph{AttriNet: Learning Mid-Level Features for Human Activity Recognition with Deep Belief Networks}", International Conference on Mobile Systems, Applications, and Services (MobiSys 2016)
% \end{enumerate}









\end{rSection}
\begin{rSection}{Teaching}
\begin{rSubsection}{How to write Fast Code}{Spring 2015, Spring 2016}{}{}
\item Lead TA for openMP and SIMD sections of 18645-How To Write Fast Code course 
\end{rSubsection}
\end{rSection}

\newpage
\begin{rSection} {Miscellaneous Academia}
\item Reviewer for NAACL 2018.
\end {rSection}

\begin{rSection}{Contests and Awards}
\begin{rSubsection}{}{}{}{}
\item Emirates Travel Hackathon 2013 Winner in the best windows phone app category.
\item NestGSV Hackathon 2014 Vuzix Glass Winner.
\end{rSubsection}
\end{rSection}

\begin{rSection}{Other Activities}
\begin{rSubsection}{}{}{}{}
\item Member of Eta Kappa Nu since Jan 2011.
\item Chair-person for ECE Graduate Organization, Silicon Valley Branch from Aug 2012 to July 2013.
\item Committee Member of Master's Advisory Council at Carnegie Mellon University from Jan 2011 to Dec 2011.
%\item External reviewer for International Conference on Fuzzy System and Data Mining (FSDM 2015)
\item Executive Committee Member for IEEE Student Branch in Vishwakarma Institute of Technology from 2008-2009.
\item Member of Editorial Board for the college magazine at Vishwakarma Institute of Technology from 2008-2010.

\end{rSubsection}
\end{rSection}
\begin{rSection}{Personal Attributes}
\begin{rSubsection}{}{}{}{}
\item Programming Languages: Python, C, C++, Java, Matlab, CUDA, and OpenCL.
\item Proficient in the usage of Kaldi Speech recognition toolkit.
\item Intermediate expertise in usage of MXNet, TensorFlow.
\item Fluent in English, Tamil, Hindi, and Marathi.
\item I enjoy long distance running, hiking, reading, soccer, and playing the guitar, synthesizer, and ukelele.
\end{rSubsection}
\end{rSection}

%----------------------------------------------------------------------------------------
%	TECHNICAL STRENGTHS SECTION
%----------------------------------------------------------------------------------------

%\begin{rSection}{Technical Strength}

%\begin{tabular}{ @{} >{\bfseries}l @{\hspace{6ex}} l }
%Computer Languages & Prolog, Haskell, AWK, Erlang, Scheme, ML \\
%Protocols \& APIs & XML, JSON, SOAP, REST \\
%Databases & MySQL, PostgreSQL, Microsoft SQL \\
%Tools & SVN, Vim, Emacs
%\end{tabular}

%\end{rSection}

%----------------------------------------------------------------------------------------
%	EXAMPLE SECTION
%----------------------------------------------------------------------------------------

%\begin{rSection}{Section Name}

%Section content\ldots

%\end{rSection}

%----------------------------------------------------------------------------------------

\end{document}
